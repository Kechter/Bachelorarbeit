\documentclass[oneside]{ausarbeitung}
\bibliography{latexlit}

% ----------------------------------------------------------------------

\begin{document}

%--- Sprachauswahl
% Erlaubte Werte:
%   \selectlanguage{english}
%   \selectlanguage{ngerman}
\selectlanguage{ngerman}

%--- Art der Arbeit
% Erlaubte Werte:
%   \Praxissemesterbericht
%   \Projektbericht
%   \Bachelorarbeit
%   \Seminararbeit
%   \Masterarbeit

\Bachelorarbeit

%--- Studiengang:
% Erlaubte Werte:
%   \Informatik
%   \Elektronik
%   \DataScience
\Informatik

\title{Component-Based Web Development: Eine Revolution in der Wiederverwendbarkeit von Webkomponenten?}

\author{Jonathan Kechter}
\matrikelnr{83701}

%--- Ist der Erstbetreuer (\examinerA) an der Hochschule ein Professor?
% Erlaubte Werte:
%   \examinerIsAProfessortrue   % Ja
%   \examinerIsAProfessorfalse  % Nein
\examinerIsAProfessortrue   % Ja

%--- Betreuer
\examinerA{Prof. Dr. Carsten Lecon}
\examinerB{Sebastian Stigler}

%--- Einreichungsdatum
\date{02. April 2025}

%--- Angaben zur Firma
% Auskommentieren, wenn die Arbeit nicht bei einer ext. Firma gemacht wurde.


%--- Angaben zum Betreuer bei dieser Firma


%--- Titelseite Anzeigen
\maketitle
\cleardoublepage

%---
\pagenumbering{roman}
\setcounter{page}{1}

%--- Firmendaten Anzeigen
% Auskommentieren, wenn die Arbeit nicht bei einer ext. Firma gemacht wurde.
\makeworkplace
\cleardoublepage

%--- Eidesstattliche Erklärung anzeigen
\makeaffirmation
\cleardoublepage

%--- Sperrvermerk (Funktioniert nur bei externen Bachelor- oder Masterarbeiten.)
%\makeconfidentialclause
\cleardoublepage

%---
\begin{abstract}
  Ziel der Kurzfassung ist es, einen (eiligen) Leser zu informieren, so 
  dass dieser entscheiden kann, ob der Bericht für ihn hilfreich ist oder 
  nicht (neudeutsch: Management Summary). Die Kurzfassung gibt daher eine 
  kurze Darstellung

  \begin{itemize}
    \item des in der Arbeit angegangenen Problems
    \item der verwendeten Methode(n)
    \item des in der Arbeit erzielten Fortschritts.
  \end{itemize}

  Dabei sollte nicht auf die Struktur der Arbeit eingegangen werden, also 
  Kapitel~\ref{cha:grundlagen} etc. denn die Kurzfassung soll ja gerade 
  das Wichtigste der Arbeit vermitteln, ohne dass diese gelesen werden muss.
  Eine Kapitelbezogene Darstellung sollte sich in Kapitel~%
  \ref{cha:einleitung} unter Vorgehen befinden.

  Länge: Maximal 1 Seite.
\end{abstract}
%-----------------------------------------------------------------------
\cleardoublepage
\tableofcontents

%---
\listoffigures

%---
\listoftables

%---
\lstlistoflistings

%---
\listofabbreviations
\begin{acronym}[Bsp.]  % Längstes Kürzel in der nachfolgenden
                       % Liste um die Breite der Spalte für die
                       % Abkürzungen zu bestimmen.

%% Eintrag: \acro{Referenzname}[Kürzel]{Langform}
%% Im Text wird die Abkürzung dann mit \ac{Referenzname} benutzt.
\acro{rup}[RUP]{Rational Unified Process}
%\acro{bsp}[Bsp.]{Beispiel}
\end{acronym}
%---


\cleardoublepage
\pagenumbering{arabic}
\setcounter{page}{1}

% ----------------------------------------------------------------------
\chapter{Einleitung}
\label{cha:einleitung}

Die Einleitung dient dazu, beim Leser Interesse für die Inhalte der Bachelorarbeit zu wecken, die behandelten Probleme aufzuzeigen und die zu ihrer Lösung entwickelten Konzepte zu beschreiben.

\section{Motivation}
\label{sec:motivation}

In der Motivation wird dargestellt, welche Bedeutung die im Rahmen der Arbeit entwickelten Lösungen haben.

\section{Problemstellung und -abgrenzung}
\label{sec:problemstellung}

Die Problemstellung dient dazu, das zu lösende Problem klar zu definieren und abzugrenzen. 

\section{Ziel der Arbeit}
\label{sec:ziel}

Mit dem Ziel der Arbeit wird der angestrebte Lösungsumfang festgelegt.

\section{Vorgehen}
\label{sec:vorgehen}

Nachdem mit Problemstellung und Ziel gewissermaßen Anfangs- und Endpunkt beschrieben sind, wird hier der zur Erreichung des Ziels eingeschlagene Weg vorgestellt.

% ---
\chapter{Grundlagen}
\label{chap:grundlagen}

% Einführung in das Kapitel
- Beschreibung der technologischen Grundlagen der Arbeit  
- Komponentenbasierter Ansatz mit verschiedenen Frameworks  
- Zentrales Backend für alle Frontends  

\section{TypeScript}
- Typisierte Obermenge von JavaScript  
- Entwickelt von Microsoft  
- Statische Typisierung, bessere Code-Wartung  
- Nutzung für Backend und Frontend  
- Gemeinsame TypeScript-Module für zentrale Backend-Funktionalitäten  

\section{Supabase}
- Open-Source-Alternative zu Firebase  
- PostgreSQL-Datenbank mit eingebauter Authentifizierung  
- Backend für Benutzerverwaltung und Datenbankoperationen  
- TypeScript-Schnittstelle für zentrale Nutzung durch alle Frontends  

\section{Frontend-Technologien}

\subsection{Angular}
- TypeScript-basiertes Framework von Google  
- Komponentenbasierte Architektur  
- Features: Dependency Injection, Routing, Reaktivität  
- Nutzung von \texttt{app.config.ts} und \texttt{app.routes.ts}  

\subsection{React}
- JavaScript-Bibliothek von Meta  
- Deklarativer Ansatz, virtueller DOM  
- Nutzung von Funktionskomponenten und Hooks  
- Fokus auf Modularität und Wiederverwendbarkeit  

\subsection{Web Components}
- Plattformunabhängige UI-Komponenten  
- Standardisiert mit Shadow DOM und Custom Elements  
- Keine Abhängigkeit von bestimmten Frameworks  

\subsection{jQuery}
- Älteste und weit verbreitete JavaScript-Bibliothek  
- Erleichtert DOM-Manipulation und Event-Handling  
- Kein komponentenbasierter Ansatz  
- Analyse von Modularität und Wartbarkeit im Vergleich zu modernen Frameworks  

\section{Zusätzliche Werkzeuge}

\subsection{LaTeX}
- Dokumentenerstellungssystem für wissenschaftliche Arbeiten  
- Ermöglicht strukturierte und professionelle Formatierung  
- Nutzung für die Erstellung der Bachelorarbeit  

\subsection{Git}
- Versionskontrollsystem zur Nachverfolgung von Code-Änderungen  
- Verteilte Versionsverwaltung für kollaborative Softwareentwicklung  
- Sicherung und Wiederherstellung von Projektständen  

\subsection{Sourcetree}
- Grafische Benutzeroberfläche für Git  
- Erleichtert das Arbeiten mit Branches, Commits und Merges  
- Alternative zu Kommandozeilen-Git für bessere Übersichtlichkeit  

\subsection{VS Code}
- Leichtgewichtige, erweiterbare Entwicklungsumgebung von Microsoft  
- Unterstützung für TypeScript, JavaScript, LaTeX und viele weitere Sprachen  
- Integrierte Git-Unterstützung, Debugging und Extensions  
- Nutzung als primäre Entwicklungsumgebung im Projekt  

\section{Zusammenfassung}
- Beschreibung der verwendeten Technologien  
- TypeScript als Programmiersprache für das Backend  
- Vergleich verschiedener Frontend-Technologien hinsichtlich Modularität und Wiederverwendbarkeit  
- Supabase als zentrale Lösung für Datenbank und Authentifizierung  
- Nutzung von Git und Sourcetree für Versionskontrolle  
- LaTeX für die Dokumentation der Arbeit  
- VS Code als zentrale Entwicklungsumgebung  



%---
\chapter{Problemanalyse}
\label{cha:problemanalyse}

Die Analyse des zu lösenden Problems ist Grundlage für jedes ingenieurmäßige Vorgehen. 

%---
\chapter{Lösungskonzept}
\label{cha:loesungskonzept}

Auf der Basis der im vorangegangenen Kapitel erstellten Problemanalyse wird ein Lösungskonzept erarbeitet.

%---
\chapter{Implementierung}
\label{cha:implementierung}

In diesem Kapitel wird die konkrete Implementierung des entwickelten Lösungskonzepts beschrieben.

%---
\chapter{Evaluierung}

Aufgabe des Kapitels Evaluierung ist es, in wie weit die Ziele der Arbeit erreicht wurden.

%---
\chapter{Zusammenfassung und Ausblick}
\label{cha:zusammenfassung}

\section{Erreichte Ergebnisse}
\label{sec:ergebnisse}

\section{Ausblick}
\label{sec:ausblick}

\subsection{Erweiterbarkeit der Ergebnisse}
\label{sub:erweiterbarkeit}

\subsection{Übertragbarkeit der Ergebnisse}
\label{sub:uebertragbarkeit}

%-----------------------------------------------------------------------
\appendix

%---
\printbibliography[heading=bibintoc]

%---
\chapter{Anhang A}

%---
\chapter{Anhang B}

\end{document}
